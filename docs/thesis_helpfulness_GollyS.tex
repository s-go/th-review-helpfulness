\documentclass[
    %german,
    a4paper,%
    %11pt,%
    12pt,%
    oneside,%
    %twoside,%
    %titlepage,%
    %liststotoc,%
    toc=bibliography,
    %bibtotoc,%
    %headinclude,%
    %draft,
    final,
    %pointlessnumbers,%
    %fleqn,% mathematische Gleichungen linksbündig statt zentriert
]{scrartcl}

%\usepackage{polyglossia} % (neue) deutsche Beschriftungen und Silbentrennung
%\setdefaultlanguage[spelling=new]{german}
%\usepackage{times} % Nimbus Roman statt CM Serif
\usepackage{lmodern} % Latin Modern (in T1) statt CM
%%\usepackage[T1]{fontenc} % T1-Kodierung: Umlaute als *eine* Glyphe
%%\usepackage[utf8]{inputenc}
\usepackage{fontspec}
\setsansfont{Myriad Pro}
\setmainfont{Adobe Garamond Pro}

%%%%%%%%%%%%%%%% Seitenspiegel, Typografie %%%%%%%%%%%%%%%%%%%
%\usepackage{pdflscape} % Querformat: \begin{landscape}
\usepackage{geometry} % Seitenränder selbst bestimmen
\geometry{a4paper,%
          top=18mm,%
          left=20mm,%
          right=20mm,% ohne Marginalien: 20mm - mit Marginalien: 45mm
          bottom=22mm,%
          headsep=10mm,%
          footskip=12mm,%
         }
\setlength{\parindent}{0pt} % kein Einruecken bei Absatzbeginn
\setlength{\parskip}{8pt} % Absaetze durch Abstand kennzeichnen (1 Zeichenhoehe)

% Space between footnote mark and text
\usepackage[hang]{footmisc}
\setlength{\footnotemargin}{1em}

\usepackage{caption}
\captionsetup{
  labelfont=bf,
}

\usepackage{threeparttable}
\renewcommand{\TPTminimum}{\linewidth}

\usepackage{makecell} % \thead{} \makecell{}
\renewcommand\theadalign{cb}
\renewcommand\theadfont{\bfseries}

\usepackage{ulem} % durchgestrichener Text: \sout{}
\usepackage[usenames,dvipsnames,svgnames,table]{xcolor} % schönere Farben, z. B. RawSienna
\usepackage{enumerate} % Aufzählungsstil anpassen, z. B. {enumerate}[a)] – \setcounter{enumi}{4}

\usepackage{natbib} % \bibitem[Guevara(2010)]{Guevara2010} - \citet[5]{Guevara2010}
\bibpunct{(}{)}{;}{a}{}{,} % Interpunktion in Zitaten
\setcitestyle{notesep={: }} % Doppelpunkt zwischen Jahr und Seitenzahl

\usepackage{tocstyle}
\newtocstyle[KOMAlike][leaders]{alldotted}{}
\usetocstyle{alldotted}

%%%%%%%%%%%%%%%% Quelltext-Satz %%%%%%%%%%%%%%%%%%%
\usepackage{textcomp} % Text Companion fonts (für einfache Anführungszeichen)
\usepackage{listings} % Umgebung lstlisting; \lstinline$...$
\lstset{
	language=,                % the language of the code
	basicstyle=\ttfamily\small,
	xleftmargin=2em,
	xrightmargin=2em,
	captionpos=b,
	abovecaptionskip=.5em,
	commentstyle=\color{OliveGreen},% sets comment style
	tabsize=3,                      % sets default tabsize
	breaklines=true,                % sets automatic line breaking
	breakatwhitespace=true,         % sets if automatic breaks should only happen at whitespace
	showspaces=false,               % show spaces adding particular underscores
	showstringspaces=false,         % underline spaces within strings
	escapechar=§,                   % escapes to LaTeX
	columns=flexible,               % columns=fixed / columns=flexible / columns=fullflexible
	upquote=true,                   % straight quotes
	literate={ö}{{\"o}}1            % national characters:
			 {ä}{{\"a}}1            %   *{replace}{replacement text}{length in output}
			 {ü}{{\"u}}1            %   * (optional): not in delimited text (strings, comments, ...) 
			 {Ö}{{\"O}}1
			 {Ä}{{\"A}}1
			 {Ü}{{\"U}}1
			 {ß}{{\ss}}2
	}

%%%%%%%%%%%%%%%%%%% Linguistik-Pakete %%%%%%%%%%%%%%%%%%%%%%%%
\usepackage{mathtools} % includes amsmath, adds some nice fixes
\usepackage{amssymb} % für \varnothing
%\usepackage{semantic} % für |[ |]
%\usepackage{qtree} % qtree \Tree [. ] [. ] \qroof{}.
%\usepackage[x11names, rgb]{xcolor} % für dot2tex
%\usepackage{tikz} % für dot2tex
%\usetikzlibrary{arrows,shapes} % für dot2tex
         %%% unbedingt nach tikz-Paketen laden %%%
%\usepackage{gb4e} % Beispiel-Umgebung: \begin{exe} \ex \begin{xlist}
%\usepackage{avm}             % für AVMs
%\avmfont{\sc}                % allgemeine AVM-Schriftart
%\avmvalfont{\it}             % Schriftart für Werte
%\avmsortfont{\footnotesize\it} % Schriftart für Sort Labels 

%%%%%%%%%%%%%%%%%%% hyperref %%%%%%%%%%%%%%%%%%%%%%%%
\usepackage[hidelinks]{hyperref} % letzter Paketaufruf!
\makeatletter % changes the catcode of @ to 11
\AtBeginDocument{
  \hypersetup{ % hyperref: \title und \author in PDF-Eingenschaften übernehmen
    pdftitle = {\@title},
    pdfsubject = {\@subject},
    pdfauthor = {\@author}
  }
}
\makeatother % changes the catcode of @ back to 12

%%%%%%%%%%%%%%%%%%%%%%% Titel %%%%%%%%%%%%%%%%%%%%%%%%%%
\subject{Bachelor Thesis}
\title{Investigating the \\Impact of Discourse Relations\\ on Review Helpfulness}
\subtitle{}
%\setkomafont{author}{\normalsize}
\setkomafont{date}{\normalsize}
\setkomafont{publishers}{\normalsize}
\author{Sebastian Golly\\ {\normalsize (761737)}}
\date{\today}

% Double line spacing for text body, but not for titles
\usepackage[doublespacing]{setspace}
\addtokomafont{disposition}{\linespread{1}}

% No extra line spacing between list items
\usepackage{enumitem}
\setlist{nosep}

\begin{document}
%%%%%%%%%%%%%%%%%%%%%%%%%%%%%%%%%%%%%%%%%%%%%%%%%%%%%%%%%%%%%%
\maketitle

\vfill

\paragraph{Abstract}

% TODO: Write abstract
Previous studies...
\\[3em]

\vfill

\begin{center}
University of Potsdam\\[1.5em]
Department of Linguistics
\end{center}

\thispagestyle{empty}
\newpage

%%%%%%%%%%%%%%%%%%%%%%%%%%%%%%%%%%%%%%%%%%%%%%%%%%%%%%%%%%%%%%

\section{Introduction}



\section{Previous Work}
\label{sec:previous-work}



\section{Goal of this Study}
\label{sec:goal}




\section{Method}
\label{sec:method}


\subsection{Data}


\subsection{Learning Task}


\subsection{Features}



\subsection{Experimental Setup}

\begin{table}[h!]
	\centering
	
	\caption{Evaluation results of a linear SVR model using different variants of discourse-relation features}
	\label{tab:performance}
	
	\begin{threeparttable}
	\renewcommand{\arraystretch}{1.5}
	\makebox[\linewidth]{
		\begin{tabular}{lcc}
		\thead{Feature} & \thead{Electronics\\ Pearson's \textit{r}\tnote{a}} & \thead{Books\\ Pearson's \textit{r}\tnote{a}} \\ \hline
		\lstinline|REL-CNT| & 0.211 (± 0.040) & 0.167 (± 0.058) \\ \hline
		\lstinline|REL-IPT| & 0.086 (± 0.040) & 0.125 (± 0.064) \\ \hline
		\lstinline|REL-PRS| & 0.279 (± 0.056) & 0.237 (± 0.084) \\ \hline
		\end{tabular} 
	}
	
	\begin{tablenotes}
	\centering
	\footnotesize
	\item[a] 95\% confidence bounds are calculated using 10-fold cross-validation.
	\end{tablenotes}
	
	\end{threeparttable}

\end{table}

\section{Results}
\label{sec:results}

Table~\ref{tab:feature-weights-electronics} shows ...

\begin{table}[h!]
	\centering
	
	\caption{Ranking of discourse-relation types according to their feature-weight coefficients in the model trained on electronics reviews}
	\label{tab:feature-weights-electronics}
	
	\begin{threeparttable}
	\renewcommand{\arraystretch}{1.5}
	\makebox[\linewidth]{
		\begin{tabular}{clr}
		\textbf{Rank} & \textbf{PDTB Sense Tag} & \textbf{Coefficient}\tnote{\textit{a}} \\ \hline
		1 & Expansion.Conjunction & 0.048 \\ \hline
		2 & Temporal.Synchrony  & 0.023 \\ \hline
		3 & Comparison.Contrast\tnote{\textit{b}} & 0.018 \\ \hline
		4 & Contingency.Condition\tnote{\textit{b}}  & 0.012 \\ \hline
		5 & Contingency.Cause\tnote{\textit{b}}  & 0.011 \\ \hline
		6 & Comparison.Concession\tnote{\textit{b}} & 0.009 \\ \hline
		7 & Expansion.Restatement  & 0.008 \\ \hline
		8 & Expansion.List  & 0.004 \\ \hline
		9 & Temporal.Asynchronous    & 0.001 \\ \hline
		10 & Expansion.Instantiation & -0.000 \\ \hline
		11 & Expansion.Exception  & -0.001 \\ \hline
		12 & Expansion.Alternative  & -0.003 \\ \hline
		\end{tabular} 
	}
	
	\begin{tablenotes}
	\centering
	\footnotesize
	\item[\textit{a}] Coefficients calculated using 10-fold cross-validation.
	\item[\textit{b}] Including \textit{Pragmatic} type.
	\end{tablenotes}
	
	\end{threeparttable}
\end{table}

\begin{table}[h!]
	\centering
	
	\caption{Ranking of discourse-relation types according to their feature-weight coefficients in the model trained on book reviews}
	\label{tab:feature-weights-books}
	
	\begin{threeparttable}
	\renewcommand{\arraystretch}{1.5}
	\makebox[\linewidth]{
		\begin{tabular}{clr}
		\textbf{Rank} & \textbf{PDTB Sense Tag} & \textbf{Coefficient}\tnote{\textit{b}} \\ \hline
		1 & Expansion.Conjunction & 0.093 \\ \hline
		2 & Temporal.Synchrony  & 0.028 \\ \hline
		3 & Comparison.Contrast\tnote{\textit{b}} & 0.017 \\ \hline
		4 & Expansion.Instantiation & 0.015 \\ \hline
		5 & Temporal.Asynchronous    & 0.013 \\ \hline
		6 & Comparison.Concession\tnote{\textit{b}} & 0.007 \\ \hline
		7 & Expansion.Restatement  & 0.000 \\ \hline
		8 & Contingency.Condition\tnote{\textit{b}}  & -0.006 \\ \hline
		9 & Contingency.Cause\tnote{\textit{b}}  & -0.006 \\ \hline
		10 & Expansion.Alternative  & -0.025 \\ \hline
		11 & Expansion.Exception  & -0.033 \\ \hline
		12 & Expansion.List  & -0.041 \\ \hline
		\end{tabular} 
	}
	
	\begin{tablenotes}
	\centering
	\footnotesize
	\item[\textit{a}] Coefficients calculated using 10-fold cross-validation.
	\item[\textit{b}] Including \textit{Pragmatic} type.
	\end{tablenotes}
	
	\end{threeparttable}
\end{table}

\begin{table}[h!]
	\centering
	
	\caption{Comparison of discourse-relation ranks between models trained on electronics vs. book reviews}
	\label{tab:rank-comparison}
	
	\begin{threeparttable}
	\renewcommand{\arraystretch}{1.5}
	\makebox[\linewidth]{
		\begin{tabular}{lcc}
		\textbf{PDTB Sense Tag} & \textbf{Electronics Rank} & \textbf{Books Rank} \\ \hline
		Expansion.Conjunction & 1 & 1 \\ \hline
		Temporal.Synchrony  & 2 & 2 \\ \hline
		Comparison.Contrast\tnote{\textit{a}} & 3 & 3 \\ \hline
		Contingency.Condition\tnote{\textit{a}}  & \textbf{4} & \textbf{8} \\ \hline
		Contingency.Cause\tnote{\textit{a}}  & \textbf{5} & \textbf{9} \\ \hline
		Comparison.Concession\tnote{\textit{a}} & 6 & 6 \\ \hline
		Expansion.Restatement  & 7 & 7 \\ \hline
		Expansion.List  & \textbf{8} & \textbf{12} \\ \hline
		Temporal.Asynchronous    & \textbf{9} & \textbf{5} \\ \hline
		Expansion.Instantiation & \textbf{10} & \textbf{4} \\ \hline
		Expansion.Exception  & 11 & 11 \\ \hline
		Expansion.Alternative  & \textbf{12} & \textbf{10} \\ \hline
		\end{tabular} 
	}
	
	\begin{tablenotes}
		\centering
		\footnotesize
		\item[\textit{a}] Including \textit{Pragmatic} type.
	\end{tablenotes}
	
	\end{threeparttable}
\end{table}

\section{Discussion}
\label{sec:discussion}

\subsection{Implications}



\subsection{Limitations}



\subsection{Future Work}



\section{Conclusion}
\label{sec:conclusion}



\vfill


\begin{center}
Both code and data used for this study are freely available for research purposes at \url{https://github.com/s-go/th-review-helpfulness}.
\end{center}

\newpage
\begin{thebibliography}{9}



\end{thebibliography}

\end{document}
